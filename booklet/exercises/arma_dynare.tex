\section[ARMA(1,1) simulation]{ARMA(1,1) simulation\label{ex:ARMASimulation}}
Consider the ARMA(1,1) model:
$$ x_t - \theta x_{t-1} = \varepsilon_t - \phi \varepsilon_{t-1}$$
where $\varepsilon_t \sim N(0,1)$.

\begin{enumerate}
	\item Compute the non-stochastic steady-state with pen and paper, i.e. what is the value of $x_t$ if $\varepsilon_t = 0$ for all $t$?
	\item Write a Dynare mod file for this model:
	\begin{itemize}
		\item $x$ is the endogenous variable, $\varepsilon$ the exogenous variabel, and $\theta$ and $\phi$ are the parameters.
		\item Set $\theta=\phi=0.4$.
		\item Write either a \texttt{steady\_state\_model} or \texttt{initval} block and compute the steady-state.
		\item Start at the non-stochastic steady-state and simulate 200 data points using a \texttt{shocks} block 
		and the \texttt{stoch\_simul} command. Drop the first 50 observations and plot both $x$ as well as $\varepsilon$.
		\item Try out different values for $\theta$ and $\phi$. What do you notice?
	\end{itemize}
	\item Redo the exercise in MATLAB without using Dynare.	
\end{enumerate}


\begin{solution}\textbf{Solution to \nameref{ex:ARMASimulation}}
\ifDisplaySolutions

\fi
\newpage
\end{solution}