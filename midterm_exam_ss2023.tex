% !TeX encoding = UTF-8
% !TeX spellcheck = en_US
\documentclass{article}
%%%%%%%%%%%%%%%%%%%%%%%%%%%%%%%%%%%%%%%%%%%%%%%%%%%%%%%%%%%%%%%%%%%%%%%%%%%%%%%%%%%%%%%%%%%%%%%%%%%%%%%%%%%%%%%%%%%%%%%%%%%%%%%%%%%%%%%%%%%%%%%%%%%%%%%%%%%%%%%%%%%%%%%%%%%%%%%%%%%%%%%%%%%%%%%%%%%%%%%%%%%%%%%%%%%%%%%%%%%%%%%%%%%%%%%%%%%%%%%%%%%%%%%%%%%%
\usepackage[a4paper,top=2cm]{geometry}
\usepackage{amssymb,amsmath,amsfonts}
\usepackage[english]{babel}
\usepackage[a4paper]{geometry}
\usepackage{enumitem}
\usepackage{booktabs}
\usepackage{csquotes}
\usepackage{url}
\usepackage{graphicx}
%\parindent0mm
%\parskip1.5ex plus0.5ex minus0.5ex
\usepackage[numbered,framed]{matlab-prettifier}

\begin{document}
	
\title{Computational Macroeconomics\\Midterm Exam}
\author{Willi Mutschler}
\date{Summer 2023\\~\\Version 1.0.1}
\maketitle

\section*{General information}
\begin{itemize}
\item Answer \textbf{all} of the following \textbf{two} exercises in English.
\item All assignments will be given the same weight in the final grade.
\item Hand in your solutions before Friday June, 09 2022 at 3pm.
\item The solution files should contain your executable (and commented) Dynare files, MATLAB functions and script files
  as well as all additional documentation as \texttt{pdf}, not \texttt{doc} or \texttt{docx}.
Your \texttt{pdf} files may also include scans or pictures of handwritten notes.
\item Please e-mail ALL the solution files to \url{willi.mutschler@uni-tuebingen.de}.
\item I will confirm the receipt of your work also by email (typically within the hour). If not, please resend it to me.
\item All students must work on their own, please also give your student ID number.
\item It is advised to regularly check Ilias and your emails in case of urgent updates.
\item If there are any questions, do not hesitate to contact Willi Mutschler.
\end{itemize}

\section*{Changelog}
Version 1.0.1:
\begin{itemize}
	\item Fixed typo in exercise 1.9 (elasticity is $\nu_t$ and not $\eta_t$)
	\item Fixed typo in title of paper for exercise 2.
\end{itemize}

\newpage

\section[Monopolistic competition and irreversible investments]{Monopolistic competition and irreversible investments\label{ex:RBCModelMonCompIrrInvest}}

\begin{center} \Large \textbf{Model Description} \end{center}

\paragraph{Households: Utility}
The economy is assumed to be inhabited by a representative family with identical members.
The household's preferences are defined over consumption $c_t$
  and labor effort $h_t$.
The representative household maximizes present as well as expected future utility
\begin{align}
	\max E_t \sum_{s=0}^{\infty} \beta^{s} \left( \frac{c_{t+s}^{1-\sigma_c}}{1-\sigma_c} - \chi_h \frac{h_{t+s}^{1+\sigma_h}}{1+\sigma_h} \right)\label{eq:RBCMonopIrrInv.UtilityLifetime}
\end{align}
where $E_t$ is the expectation operator conditional on information at time $t$,
  $\beta <1$ denotes the discount factor,
  $\sigma_c$ and $\sigma_h$ are elasticities and $\chi_h$ the consumption weight.

\paragraph{Households: Capital Accumulation}
The household owns the (end of period) capital stock $k_t$ which evolves according to
\begin{align}
k_t = (1-\delta)k_{t-1} + i_t \label{eq:RBCMonopIrrInv.CapitalAccumulation}
\end{align}
where $\delta$ is the depreciation rate.

\paragraph{Households: Irreversible Investment}
An occasionally binding constraint is imposed that prevents investment $i_t$ from falling below a fraction $\omega$
  of its value in the non-stochastic steady-state (denoted by $i$):
\begin{align}
i_t \geq \omega \cdot i \label{eq:RBCMonopIrrInv.IrrInvest}
\end{align}

\paragraph{Households: Budget}
Capital is rented to the intermediate firms at a nominal rate of $R^k_{t}$ which the household takes as given when forming optimal plans.
Similarly, in each period the household takes the nominal wage $W_t$ as given and supplies perfectly elastic labor service $h_t$ to the firm sector.
In return she receives nominal labor income $W_t h_t$.
All firms are owned by the household so that nominal profits and dividends from firms in the final good sector, $ Div^{Fin}_t$,
  as well as from each firm $f\in[0,1]$ in the intermediate goods sector, $\int_0^1 {Div}^{Int}_t(f)df$,
  are received by the household.
Income and wealth are used to finance consumption and investment expenditures, both are quoted at price level $P_t$.
In total this defines the \emph{nominal} budget constraint of the household
\begin{align}
P_t c_t + P_t i_t =  W_t h_t + R^k_t k_{t-1} + Div^{Fin}_t + \int_0^1 Div^{Int}_t(f) df \label{eq:RBCMonopIrrInv.BudgetNominal}
\end{align}
In what follows, let lower case letters denote real variables, i.e.\
\begin{align*}
w_t=\frac{W_t}{P_t},~~ r^k_t = \frac{R^k_t}{P_t},~~ div^{Fin}_t = \frac{Div^{Fin}_t}{P_t},~~ div^{Int}_t = \frac{Div^{Int}_t}{P_t}
\end{align*}

\paragraph{Final Good Firm (Retail Sector): Profit Maximization}
The economy is populated by a continuum of firms indexed by $f \in [0,1]$ that produce differentiated goods $y_t(f)$ in monopolistic competition.
The technology for transforming these intermediate goods into the final output good $y_t$ that can be used for consumption and investment is given by:
\begin{align}
y_t = \left[\int\limits_0^1 y_t(f)^{\frac{\nu_t-1}{\nu_t}}df\right]^{\frac{\nu_t}{\nu_t-1}} \label{eq:RBCMonopIrrInv.Firms.Aggregator}
\end{align}
where $\nu_t$ is the time-varying elasticity of substitution between differentiated goods.


\paragraph{Intermediate Goods Firms (Wholesale Sector): Profit Maximization}
Intermediate firm $f\in[0,1]$ uses the following production function to produce their differentiated good
\begin{align}
y_t(f) = a_t (k_{t-1}(f))^\alpha (n_t(f))^{1-\alpha} \label{eq:RBCMonopIrrInv.IntermediateFirms.ProductionFunction}
\end{align}
where $a_t$ denotes the common technology level available to all firms.
Firms face perfectly competitive factor markets for renting capital $k_{t-1}(f)$ and hiring labor $n_t(f)$ with $\alpha$ being a productivity parameter.
Nominal profits of firm $f$ are equal to revenues from selling its differentiated good at self-determined price $P_t(f)$
  minus costs from hiring labor at real wage $w_t$ and real renting rate of capital $r^k_t$:
\begin{align}
{Div}^{Int}_t(f) = P_t(f) y_t(f) - P_t w_t n_t(f) - P_t r^k_t k_{t-1}(f) \label{eq:RBCMonopIrrInv.Firms.Profits}
\end{align}
The objective of the firm is to choose contingent plans for $P_t(f)$, $n_t(f)$ and $k_{t-1}(f)$
  so as to maximize the present discounted value of nominal dividend payments given by
\begin{align*}
E_t \sum_{s=0}^{\infty} \beta^s \frac{\lambda_{t+s+1}}{\lambda_{t+s}} Div^{Int}_{t+s}(f)
\end{align*}

\paragraph{Exogenous Variables}
The level of technology $a_t$ and the elasticity $\nu_t$ evolve according to
\begin{align}
\log{a_t} &= \rho_a \log{a_{t-1}} + \varepsilon_{a,t} \label{eq:RBCMonopIrrInv.LoM.TFP}\\
\nu_t &= \bar{\nu} + \varepsilon_{\nu,t} \label{eq:RBCMonopIrrInv.LoM.Elast}
\end{align}
with persistence parameter $\rho_a$ and target value $\bar{\nu}>1$.
$\varepsilon_{a,t}$ and $\varepsilon_{\nu,t}$ are the exogenous variables.

\paragraph{Calibration}
Use the following values for the model parameters:
\begin{align*}
\beta = 0.96,
\sigma_c = 2,
\sigma_h = 2,
\delta = 0.1,
\chi_h = 1.6, 
\alpha = 0.35,
\rho_a = 0.9,
\bar{\nu} = 5,
\omega = 0.975
\end{align*}

\paragraph{Hints}
\begin{itemize}
\item A version of the model consists of 11 model equations which are given in
\eqref{eq:RBCMonopIrrInv.CapitalAccumulation},
\eqref{eq:RBCMonopIrrInv.LoM.TFP},
\eqref{eq:RBCMonopIrrInv.LoM.Elast},
\eqref{eq:RBCMonopIrrInv.LaborSupply},
\eqref{eq:RBCMonopIrrInv.EulerCapital},
\eqref{eq:RBCMonopIrrInv.KuhnTuckerInvestment},
\eqref{eq:RBCMonopIrrInv.RealMarginalCosts},
\eqref{eq:RBCMonopIrrInv.MarginalCostsAggregated},
\eqref{eq:RBCMonopIrrInv.IntermediateFirms.CapitalLaborRatioAggregated},
\eqref{eq:RBCMonopIrrInv.AggregateDemand}, and
\eqref{eq:RBCMonopIrrInv.AggregateSupply}.

\item Whenever creating plots, use a \emph{reasonable} time horizon
and scale to plot $3 \times 3=9$ model variables (of your choice).

\item In Dynare you can write the greater operand $>$ and the less-or-equal operand $<=$ just as you do in MATLAB:\\
\texttt{(i > OMEGA*steady\_state(i))} or \texttt{(i <= OMEGA*steady\_state(i))}.

\end{itemize}

\bigskip
\begin{center} \Large \textbf{Exercises} \end{center}

\begin{enumerate}

\item Explain the economic intuition behind including irreversible investments in a DSGE model.

\item Derive the intratemporal and intertemporal optimality conditions:
\begin{align}
w_t &= \chi_h h_t^{\sigma_h} c_t^{\sigma_c} \label{eq:RBCMonopIrrInv.LaborSupply}
\\
c_t^{-\sigma_c} - \mu_t &= \beta E_t \left[ c_{t+1}^{-\sigma_c} \left( r^k_{t+1} + 1-\delta \right) - (1-\delta) \mu_{t+1}\right]
\label{eq:RBCMonopIrrInv.EulerCapital}
\end{align}
where $\beta^s \mu_{t+s}$ is the Lagrange multiplier on the occasionally binding constraint \eqref{eq:RBCMonopIrrInv.IrrInvest} in period $t+s$.
Note that the complementary slackness condition
\begin{align*}
\mu_t (i_t - \omega \cdot i) &= 0
\end{align*}
can be equivalently written as
\begin{align}
(i_t > \omega \cdot i) \cdot (\mu_t) &+ (i_t \leq \omega \cdot i) \cdot (i_t - \omega \cdot i) = 0  \label{eq:RBCMonopIrrInv.KuhnTuckerInvestment}
\end{align}
Interpret these equations.

\item Show that profit maximization in the final goods sector implies:
\begin{align}
y_t(f) &= \left(\frac{P_t(f)}{P_t}\right)^{-\nu_t} y_t \label{eq:RBCMonopIrrInv.Firms.Demand}
\\
P_t y_t &= \int_{0}^{1} P_t(f) y_t(f) df \label{eq:RBCMonopIrrInv.Firms.ZeroProfit}
\end{align}
Interpret the equation.
What does this imply for real profits $div_t^{Fin}$ in the final goods sector?


\item Show that profit maximization in the intermediate good sector implies:
\begin{align}
r^k_t  &= mc_t(f) \alpha a_t \left( \frac{n_t(f)}{k_{t-1}(f)}\right)^{1-\alpha}
\label{eq:RBCMonopIrrInv.IntermediateFirms.CapitalDemand}
\\
w_t  &= mc_t(f) (1-\alpha) a_t \left(\frac{n_t(f)}{k_{t-1}(f)}\right)^{-\alpha}
\label{eq:RBCMonopIrrInv.IntermediateFirms.LaborDemand}
\\
P_t(f) &= \frac{\nu_t}{\nu_t - 1} P_t mc_t(f) \label{eq:RBCMonopIrrInv.IntermediateFirms.Price}
\end{align}
where $mc_t(f)$ is the Lagrange multiplier corresponding to constraint \eqref{eq:RBCMonopIrrInv.IntermediateFirms.ProductionFunction}.
Interpret the equations.

\item Show that all intermediate firms choose the same capital to labor ratio in production and marginal costs are the same across firms:
\begin{align}
\left(\frac{k_{t-1}(f)}{n_t(f)}\right) &= \left(\frac{w_t}{1-\alpha}\right) \left(\frac{\alpha}{r^k_t}\right) \label{eq:RBCMonopIrrInv.IntermediateFirms.CapitalLaborRatio}
\\
mc_t \equiv mc_t(f) &= \frac{1}{a_t} \left(\frac{w_t}{1-\alpha}\right)^{1-\alpha} \left(\frac{r^k_t}{\alpha}\right)^{\alpha} \label{eq:RBCMonopIrrInv.RealMarginalCosts}
\end{align}


\item Show that aggregation and market clearing implies:
\begin{align}
mc_t = \frac{\nu_t-1}{\nu_t} \label{eq:RBCMonopIrrInv.MarginalCostsAggregated}
\\
\left(\frac{k_{t-1}}{h_t}\right) &= \left(\frac{w_t}{1-\alpha}\right) \left(\frac{\alpha}{r^k_t}\right) \label{eq:RBCMonopIrrInv.IntermediateFirms.CapitalLaborRatioAggregated}
\\
\int_{0}^{1} div_t(f) df &= y_t - w_t h_t - r^k_t k_{t-1}\label{eq:RBCMonopIrrInv.IntermediateFirms.AggregateProfits}
\\
y_t &= c_t + i_t \label{eq:RBCMonopIrrInv.AggregateDemand}
\\
y_t &= a_t k_{t-1}^\alpha h_t^{1-\alpha} \label{eq:RBCMonopIrrInv.AggregateSupply}
\end{align}
Interpret the equations.

\item Derive with pen and paper the steady-state of the model, in the sense that there is a set of values for the endogenous variables that in equilibrium remain constant over time.

\item Write a Dynare mod file that computes the steady-state of this model using an \texttt{initval} block.

\item Use a perfect foresight simulation for computing the transition path for a permanent increase in the elasticity $\nu_t$ from $\bar{\nu}$ to $(\bar{\nu}+3)$.
Interpret the economic mechanisms.

\item Use a perfect foresight simulation for computing the impulse response function of the endogenous variables to a \emph{negative} TFP shock on impact of size 0.04.
Interpret the economic mechanisms.

\end{enumerate}


\newpage

\section{Simulating a war shock in a New Keynesian model}
Read the paper by Auray and Eyquem (2019, JEDC): \emph{Episodes of war and peace in an estimated open economy model}.
Particularly, focus on section 3 (The model) and section 6.3 (The macroeconomic consequences of a simulated war episode).

\paragraph{Hints}
\begin{itemize}
  \item A mod file with the model equations and a calibration for France is given in Appendix \ref{app:AurayEyquem2019.modfile}.
  \item Appendix \ref{app:AurayEyquem2019.plotfile} contains a helper function for creating plots similar to Figures 3 and 4.
  \item The war shock is in the 9th column of \texttt{oo\_.exo\_simul}.
  \item Set the \texttt{periods} option to 500.
\end{itemize}

\paragraph{Exercises}
\begin{enumerate}

\item Explain the modelling approach for adding a war into a New Keynesian model.
List all the channels a war affects the economy.
Do you find these convincing or not?

\item Replicate Figures 3 and 4 of the paper using a deterministic simulation under perfect foresight.
To this end, simulate an unanticipated war which, as soon as it occurs, is known to last for 5 years.

\item The authors claim that they are doing a \emph{MIT} shock.
Explain the concept of a MIT shock in the context of deterministic simulations.

\item Redo a sequence of 5 consecutive war shocks, but now adjusting the information set of the agents in the following manner:
\begin{itemize}
    \item Initialization, Storage for $t=0$
    \begin{itemize}
      \item Initialize the matrices \texttt{oo\_.endo\_simul} and \texttt{oo\_.exo\_simul} at the initial steady-state with no shocks
      by running \texttt{perfect\_foresight\_setup(periods=500)} after the \texttt{steady} command.
      \item \texttt{oo\_.endo\_simul} has dimension $51 \times 502$ and \texttt{oo\_.exo\_simul} has dimension $502 \times 9$.
      \item Note that the first column of \texttt{oo\_.endo\_simul} and the first row of \texttt{oo\_.exo\_simul} correspond to $t=0$ values.
      \item Store the values of the endogenous and exogenous variables for $t=0$ into \texttt{saved\_endo} and \texttt{saved\_exo}.
    \end{itemize}
    
    \item Shock occurring in $t=1$, Simulation for $t=0,...,T+1$, Storage for $t=0,1$
    \begin{itemize}
      \item Add the information of a one-time war shock arriving in period $t=1$, i.e.\
      set \texttt{oo\_.exo\_simul(2,9)=1}, and run \texttt{perfect\_foresight\_solver}.
      \item Append the values of the endogenous and exogenous variables for $t=1$,\\
        i.e.\ \texttt{oo\_.endo\_simul(:,2)} and \texttt{oo\_.exo\_simul(2,:)},
        to \texttt{saved\_endo} and \texttt{saved\_exo}.
      \item \texttt{oo\_.endo\_simul} has dimension $51 \times 502$ and \texttt{oo\_.exo\_simul} has dimension $502 \times 9$.      
      \item \texttt{saved\_endo} has dimension $51 \times 2$ and \texttt{saved\_exo} has dimension $2 \times 9$.
      \item Note that the first column of \texttt{oo\_.endo\_simul} and the first row of \texttt{oo\_.exo\_simul} correspond to $t=0$ values.
    \end{itemize}
    
    \item Shock occurring in $t=2$, Simulation for $t=1,...,T+1$, Storage for $t=0,1,2$
    \begin{itemize}
      \item Decrease the periods option by 1, i.e.\ \texttt{options\_.periods=options\_.periods-1}.
      \item Remove the values for $t=0$ from both \texttt{oo\_.endo\_simul} and \texttt{oo\_.exo\_simul},
      such that the first column of \texttt{oo\_.endo\_simul} and the first row of \texttt{oo\_.exo\_simul} correspond to $t=1$ values.
      \item Add the information of a one-time war shock arriving in period $t=2$ and run \texttt{perfect\_foresight\_solver}.
      \item Append the values of the endogenous and exogenous variables for $t=2$,\\
        i.e.\ \texttt{oo\_.endo\_simul(:,2)} and \texttt{oo\_.exo\_simul(2,:)},
        to \texttt{saved\_endo} and \texttt{saved\_exo}.
      \item \texttt{oo\_.endo\_simul} has dimension $51 \times 501$ and \texttt{oo\_.exo\_simul} has dimension $501 \times 9$.
      \item \texttt{saved\_endo} has dimension $51 \times 3$ and \texttt{saved\_exo} has dimension $3 \times 9$.
    \end{itemize}

    \item Shock occurring in $t=3$, Simulation for $t=2,...,T+1$, Storage for $t=0,1,2,3$
    \begin{itemize}
      \item Decrease the periods option by 1, i.e.\ \texttt{options\_.periods=options\_.periods-1}.
      \item Remove the values for $t=1$ from both \texttt{oo\_.endo\_simul} and \texttt{oo\_.exo\_simul},
      such that the first column of \texttt{oo\_.endo\_simul} and the first row of \texttt{oo\_.exo\_simul} correspond to $t=2$ values.
      \item Add the information of a one-time war shock arriving in period $t=3$ and run \texttt{perfect\_foresight\_solver}.
      \item Append the values of the endogenous and exogenous variables for $t=3$,\\
        i.e.\ \texttt{oo\_.endo\_simul(:,2)} and \texttt{oo\_.exo\_simul(2,:)},
        to \texttt{saved\_endo} and \texttt{saved\_exo}.
      \item \texttt{oo\_.endo\_simul} has dimension $51 \times 500$ and \texttt{oo\_.exo\_simul} has dimension $500 \times 9$.
      \item \texttt{saved\_endo} has dimension $51 \times 4$ and \texttt{saved\_exo} has dimension $4 \times 9$.
    \end{itemize}

    \item Shock occurring in $t=4$, Simulation for $t=3,...,T+1$, Storage for $t=0,1,2,3,4$
    \begin{itemize}
      \item Decrease the periods option by 1, i.e.\ \texttt{options\_.periods=options\_.periods-1}.
      \item Remove the values for $t=2$ from both \texttt{oo\_.endo\_simul} and \texttt{oo\_.exo\_simul},
      such that the first column of \texttt{oo\_.endo\_simul} and the first row of \texttt{oo\_.exo\_simul} correspond to $t=3$ values.
      \item Add the information of a one-time war shock arriving in period $t=4$ and run \texttt{perfect\_foresight\_solver}.
      \item Append the values of the endogenous and exogenous variables for $t=4$,\\
        i.e.\ \texttt{oo\_.endo\_simul(:,2)} and \texttt{oo\_.exo\_simul(2,:)},
        to \texttt{saved\_endo} and \texttt{saved\_exo}.
      \item \texttt{oo\_.endo\_simul} has dimension $51 \times 499$ and \texttt{oo\_.exo\_simul} has dimension $499 \times 9$.
      \item \texttt{saved\_endo} has dimension $51 \times 5$ and \texttt{saved\_exo} has dimension $5 \times 9$.
    \end{itemize}

    \item Shock occurring in $t=5$, Simulation for $t=4,...,T+1$, Storage for $t=0,1,2,3,4,5$
    \begin{itemize}
      \item Decrease the periods option by 1, i.e.\ \texttt{options\_.periods=options\_.periods-1}.
      \item Remove the values for $t=3$ from both \texttt{oo\_.endo\_simul} and \texttt{oo\_.exo\_simul},
      such that the first column of \texttt{oo\_.endo\_simul} and the first row of \texttt{oo\_.exo\_simul} correspond to $t=4$ values.
      \item Add the information of a one-time war shock arriving in period $t=5$ and run \texttt{perfect\_foresight\_solver}.
      \item Append the values of the endogenous and exogenous variables for $t=5$,\\
        i.e.\ \texttt{oo\_.endo\_simul(:,2)} and \texttt{oo\_.exo\_simul(2,:)},
        to \texttt{saved\_endo} and \texttt{saved\_exo}.
      \item \texttt{oo\_.endo\_simul} has dimension $51 \times 498$ and \texttt{oo\_.exo\_simul} has dimension $498 \times 9$.
      \item \texttt{saved\_endo} has dimension $51 \times 6$ and \texttt{saved\_exo} has dimension $6 \times 9$.
    \end{itemize}

    \item Combine simulations
    \begin{itemize}
      \item Re-set the periods option to 502, i.e.\ \texttt{options\_.periods=502}.
      \item Remove the values for $t=4$ and $t=5$ from both \texttt{oo\_.endo\_simul} and \texttt{oo\_.exo\_simul},
      such that the first column of \texttt{oo\_.endo\_simul} and the first row of \texttt{oo\_.exo\_simul} correspond to $t=6$ values.
      \item Note that \texttt{saved\_endo} contains the simulation for periods $t=0,1,2,3,4,5$ and \texttt{oo\_.endo\_simul} for $t=6,...,502$.
      \item Overwrite \texttt{oo\_.endo\_simul} and \texttt{oo\_.exo\_simul} such that they contain the complete simulation paths for $t=0,...,502$.      
    \end{itemize}
    \item Create the plots using the helper function.
    
\end{itemize}

\end{enumerate}

\newpage

\appendix
\section{Auray and Eyquem (2019) mod file\label{app:AurayEyquem2019.modfile}}
\lstinputlisting[style=Matlab-editor,basicstyle=\mlttfamily\scriptsize,title=\lstname]{progs/replications/Auray_Eyquem_2019/Auray_Eyquem_2019.mod}

\section{Auray and Eyquem (2019) helper function for plots\label{app:AurayEyquem2019.plotfile}}
\lstinputlisting[style=Matlab-editor,basicstyle=\mlttfamily\scriptsize,title=\lstname]{progs/replications/Auray_Eyquem_2019/Auray_Eyquem_2019_plots.m}

\end{document}
