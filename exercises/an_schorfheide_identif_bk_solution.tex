\begin{enumerate}

\item Interpretation of model equations:
\begin{itemize}
    \item Equation \eqref{eq:AS_Euler} is the Euler equation determining optimal consumption and saving decision.
    \item Equation \eqref{eq:AS_PC} is the nonlinear New Keynesian Phillips curve,
      which determines optimal price setting of the firms subject to Rotemberg price adjustment costs.
    \item Equation \eqref{eq:AS_market} is the market clearing condition;
      note that the price adjustment costs are quadratic and valued in terms of output goods.
    \item Equations \eqref{eq:AS_Taylor1} and \eqref{eq:AS_Taylor2} form the Taylor rule,
      i.e.\ monetary policy reacts to deviations of both inflation from its target
      as well as the output gap, which is defined as deviation of output from its flex-price version.
    \item Equation \eqref{eq:AS_tfpgrowth} is the law of motion of the growth rate of productivity.
    \item Equation \eqref{eq:AS_g} is the law of motion for the government spending process.
    \item Equation \eqref{eq:AS_flexy} is the output value which would occur when there are no price rigidties ($\phi=0$).
\end{itemize}

\item The mod file might look like this:
\lstinputlisting[style=Matlab-editor,basicstyle=\mlttfamily\scriptsize,title=\lstname]{progs/dynare/an_schorfheide_identif_bk.mod}

\item The first-order approximated policy function is given by:
\begin{align*}
    y_t = \bar{y} + g_y (y_{t-1} - \bar{y}) + g_u u_t
\end{align*}
where $g_y$ is a $n \times n$ matrix (i.e.\ zero columns for all variables except the state variables).
Note that due to the Blanchard-Khan conditions this is a stationary process,
  such that $E[y_t] = E[y_s] = \mu_y$ and $\Sigma_{j} \equiv COV[y_t,y_{t-j}] = COV[y_{s}, y_{s-j}]$ for all $s,t,j$.
The first moment is equal to the steady-state:
\begin{align*}
&E [y_t] = \bar{y} + g_y (E [y_{t-1}] - \bar{y}) + g_u E [u_t]
\\
&\Leftrightarrow \mu_y = \bar{y} + g_y(\mu_y - \bar{y}) + g_u \cdot 0
\\
&\Leftrightarrow \mu_y = \bar{y}
\end{align*}
Let's denote $\hat{y}_t = y_t - \bar{y}$, then the variance of $y_t$ is given by:
\begin{align*}
&\Sigma_0 = E[\hat{y}_t \hat{y}_t'] = g_y E[\hat{y}_{t-1} \hat{y}_{t-1}'] g_y' + g_u E[u_t u_t'] g_u'
\\
&\Leftrightarrow \Sigma_0 = g_y \Sigma_0 g_y' + g_u \Sigma_u g_u'
\end{align*}
This is a Lyapunov equation which you can either solve with the Kronecker product or an efficient algorithm (e.g.\ doubling).
Once we have $\Sigma_0$, the autocovariances are computed by $\Sigma_j = g_y \Sigma_{j-1}$.

\item Here is the mod file for the hawkish calibration:
\lstinputlisting[style=Matlab-editor,basicstyle=\mlttfamily\scriptsize,title=\lstname]{progs/dynare/an_schorfheide_identif_hawkish.mod}

Here is the mod file for the dovish calibration:
\lstinputlisting[style=Matlab-editor,basicstyle=\mlttfamily\scriptsize,title=\lstname]{progs/dynare/an_schorfheide_identif_dovish.mod}

Note that the steady-state and theoretical moments are (numerically) equal to each other.
Moreover, the IRFs look exactly the same (and they actually are at least numerically).
\end{enumerate}