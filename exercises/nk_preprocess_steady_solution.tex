\paragraph{Analytical steady-state} The steady-state can be computed analytically for all model variables.

In the non-stochastic steady-state we impose that all exogenous variables are equal to zero:
\begin{align*}
\varepsilon^{\mu}=0
,\quad
\varepsilon^{A}=0
,\quad
\varepsilon^{R}=0
,\quad
\varepsilon^{G}=0
\end{align*}
Accordingly, equations {\eqref{eq:NewKeynesian.LoM.PreferenceShifter}}
and {\eqref{eq:NewKeynesian.LoM.TFP}} become:
\begin{align*}
\mu = 1
,\quad
A = \overline{A}
\end{align*}
From equation~\eqref{eq:NewKeynesian.MonetaryPolicyRule} we can infer
\begin{align*}
\Pi = \overline{\Pi}
\end{align*}
The Euler equations for bonds~\eqref{eq:NewKeynesian.EulerBond},
  investment~\eqref{eq:NewKeynesian.EulerInvestment}
  and capital~\eqref{eq:NewKeynesian.EulerCapital} become in steady-state:
\begin{align*}
R = \frac{\Pi}{\beta}
,\qquad
q^{K} = 1
,\qquad
r^{K} = \frac{q^{K}}{\beta} - q^{K}(1-\delta^{K})
\end{align*}
From~\eqref{eq:NewKeynesian.ResetPriceLoM} we get:
\begin{align*}
\widetilde{p} = {\left(\frac{1 - \theta^{P} {(\Pi)}^{\epsilon^{P}-1}}{1 - \theta^{P}}\right)}^{\frac{1}{1-\epsilon^{P}}}
\end{align*}
Now we are able to evaluate~\eqref{eq:NewKeynesian.PriceDistortionLoM} in steady-state:
\begin{align*}
p^{*} = \left(\frac{1 - \theta^{P}}{1 - \theta^{P} {(\Pi)}^{\epsilon^{P}}}\right) {(\widetilde{p})}^{-\epsilon^{P}}
\end{align*}
The recursive price setting equation~\eqref{eq:NewKeynesian.IntermediateFirms.PriceSetting} in steady-state yields the auxiliary relationship:
\begin{align*}
\left(\frac{S^{1_{P}}}{S^{2_P}}\right) = \left(\frac{\epsilon^{P}-1}{\epsilon^{P}}\right) \widetilde{p}
\end{align*}
which is useful for computing the steady-state marginal costs.
To this end, evaluate~\eqref{eq:NewKeynesian.IntermediateFirms.PriceSettingSum1}
  and~\eqref{eq:NewKeynesian.IntermediateFirms.PriceSettingSum2} in steady-state
  and take the ratio:\footnote{%
Note that if we abstract from trend-inflation, i.e.\ assume price stability in steady-state, \(\overline{\Pi}=1\),
  then the expressions simplify immensely: \(\widetilde{p} = 1\), \(p^{*}=1\), and \(mc = (\epsilon^{P}-1)/\epsilon^{P}<1\).
These are useful choices for providing initial values in more difficult models.
}
\begin{align*}
{mc} &= \frac{\left(1 - \theta^{P} \beta {(\Pi)}^{\epsilon^{P}}\right)}{\left(1 - \theta^{P} \beta {(\Pi)}^{\epsilon^{P}-1}\right)} \left(\frac{S^{1_{P}}}{S^{2_{P}}}\right)
= \frac{\left(1 - \theta^{P} \beta {(\Pi)}^{\epsilon^{P}}\right)}{\left(1 - \theta^{P} \beta {(\Pi)}^{\epsilon^{P}-1}\right)} \left(\frac{\epsilon^{P}-1}{\epsilon^{P}}\right) \widetilde{p}
\end{align*}
As we have already computed steady-state values for \(r^{K}\) and \({mc}\)
  we can evaluate equation~\eqref{eq:NewKeynesian.RealMarginalCosts} in steady-state to get the steady-state real wage:
\begin{align*}
w = (1-\alpha^{K}) {\left(mc \cdot A \cdot {\left(\frac{\alpha^{K}}{r^{K}}\right)}^{\alpha^{K}} \right)}^{\frac{1}{1-\alpha^{K}}}
\end{align*}
Next we proceed just as in the RBC model
  and re-express steady-state
  capital~\eqref{eq:NewKeynesian.IntermediateFirms.CapitalLaborRatio},
  investment~\eqref{eq:NewKeynesian.CapitalAccumulation.perCapita},
  output~\eqref{eq:NewKeynesian.AggregateSupply},
  government spending~\eqref{eq:NewKeynesian.FiscalPolicyRule.spending},
  and consumption~\eqref{eq:NewKeynesian.ResourceConstraint}
  in terms of steady-state labor:
\begin{align*}
\left(\frac{k}{l^{s}}\right) &= \left(\frac{w}{1-\alpha^{K}}\right) \left(\frac{\alpha^{K}}{r^{K}}\right)
\\
\left(\frac{i}{l^{s}}\right) &= \delta^{K} \left(\frac{k}{l^{s}}\right)
\\
\left(\frac{y}{l^{s}}\right) &= {(p^{*})}^{-1} A {\left(\frac{k}{l^{s}}\right)}^{\alpha^{K}}
\\
\left(\frac{g}{l^{s}}\right) &= \overline{G_{Y}} \left(\frac{y}{l^{s}}\right)
\\  
\left(\frac{c}{l^{s}}\right) &= \left(\frac{y}{l^{s}}\right) - \left(\frac{i}{l^{s}}\right) - \left(\frac{g}{l^{s}}\right)
\end{align*}
where we made use of the fact that capital and labor market clearing implies:
\(k^{d} = k\) and \(l^{d} = l^{s}\).
Now we can manipulate the labor supply condition~\eqref{eq:NewKeynesian.LaborSupply}
  to get steady-state labor in terms of previously computed wage and consumption to labor ratio:
\begin{align*}
l^{s} = {\left( \left(\frac{w}{\chi^{L}}\right) {\left(\frac{c}{l^{s}}\right)}^{-\sigma^{C}} \right)}^{\frac{1}{\sigma^{L}+\sigma^{C}}}
\end{align*}
As we now have \(l^{s}\), we can recover all other variables from the auxiliary expressions above:
\begin{align*}
k = \left(\frac{k}{l^{s}}\right) l^{s}
,\quad
i = \left(\frac{i}{l^{s}}\right) l^{s}
,\quad
c = \left(\frac{c}{l^{s}}\right) l^{s}
,\quad
y = \left(\frac{y}{l^{s}}\right) l^{s}
,\quad
g = \left(\frac{g}{l^{s}}\right) l^{s}
,\quad
l^{d} = l^{s}
,\quad
k^{d} = k
\end{align*}
The remaining variables follow by evaluating equations {\eqref{eq:NewKeynesian.IntermediateFirms.AggregateProfits}},
{\eqref{eq:NewKeynesian.MarginalUtility}},
{\eqref{eq:NewKeynesian.IntermediateFirms.PriceSettingSum1}}
{\eqref{eq:NewKeynesian.IntermediateFirms.PriceSettingSum2}}
and {\eqref{eq:NewKeynesian.FiscalBudget}} in steady-state:
\begin{align*}
div^{M} &= y - w l^{s} - r^{K} k
\\
\lambda &= \mu c^{-\sigma^{C}}
\\
S^{2_{P}} &= \frac{y}{1- \theta^{P} \beta \Pi^{\epsilon^{P}-1}}
\\
S^{1_{P}} &= \frac{{mc} \cdot y}{1- \theta^{P} \beta \Pi^{\epsilon^{P}}}
\\
\tau &= g
\end{align*}
Lastly, all hat variables have a steady-state of 0 by definition.

\paragraph{MOD file} The following mod file contains two ways to compute the steady-state:
\begin{itemize}
  \item analytical derivation in a \texttt{steady\_state\_model} block (\texttt{USE{\_}INITVAL=false})
  \item uses numerical optimization given initial values in an \texttt{initval} block (\texttt{USE{\_}INITVAL=true})
\end{itemize}
You can use Dynare's preprocessor to choose between the two (as you cannot have both in a MOD file)
  by setting the \emph{macro variable} \texttt{USE{\_}INITVAL} appropriately.
Note that this variable is only an instruction to the preprocessor and not a variable in the model.
\lstinputlisting[style=Matlab-editor,basicstyle=\mlttfamily\scriptsize,title=\lstname]{progs/dynare/nk.mod}
