\begin{enumerate}

\item
As for all \(t\) the shocks are zero, i.e.\
  \(\varepsilon_t = 0\), we get \(x - \theta x = 0 \Leftrightarrow x = 0\),
  where \(x\) denotes the non-stochastic steady-state
  (we drop the time subscript in the original equation when computing the steady-state).

\item
\lstinputlisting[style=Matlab-editor,basicstyle=\mlttfamily\scriptsize,title=\lstname]{progs/dynare/arma11.mod}

Some notes:

\begin{itemize}

\item
Whenever the parameters are equal to each other, the simulated series for \(x_t\) is always equal to the drawn shocks \(\varepsilon_t\).

\item
Sometimes Dynare prints an error message, e.g.\ try out \(\theta=1.5\) and \(\theta=0.4\).
If you are familiar with time series models, this is due to the fact that this parameter combination provides a non-stationary process,
  i.e.\ the process explodes and does not return to the equilibrium/steady-state.
\end{itemize}

\item
This function simulates the ARMA{(1,1)} model in MATLAB:\@
\lstinputlisting[style=Matlab-editor,basicstyle=\mlttfamily\scriptsize,title=\lstname]{progs/matlab/arma11Simulate.m}
This script can be used to compare the results from Dynare with the results from MATLAB:\@
\lstinputlisting[style=Matlab-editor,basicstyle=\mlttfamily\scriptsize,title=\lstname]{progs/matlab/arma11CompareMATLABWithDynare.m}

\end{enumerate}