\begin{enumerate}

\item Compared to our baseline New Keynesian model, this model
    \begin{itemize}
    \item has the Rotemberg-pricing assumption instead of Calvo price rigidities.
    \item a linear production function instead of a Cobb-Douglas one.
    \item no capital and no investment
    \item includes a stochastic process for fiscal policy.
    \item includes a persistence term in the Taylor rule.
    \item has a unit root in total factor productivity and a shock to the growth of {TFP}.
\end{itemize}
  
\item Note that \(A_t\) has a unit root, so in model A the variables \(Y_t\), \(Y_t^*\), \(C_t\) and \(G_t\) would be trending,
  while in model B the de-trended variables \(y_t\), \(y_t^*\), \(c_t\) and \(g_t\) would be stationary.
It is important to work with the stationary model in Dynare to have a well-behaved (constant!) steady-state
  around which the perturbation solution is computed.

\item No, variables in model C behave as the log of variables in model B.
In addition, in model B, variables have distinct steady-state values depending on the parameters of the model,
  while in model C, the steady-state of all variables is zero.
If, however, we would add hat variables to model B,
  then the hat variables would be equivalent between model variants.

\item The exponential transform enables one to write variables in terms of log deviations from their steady-state:
\begin{align*}
x_t = e^{\log(x_t)} = e^{\log(x_t) - \log(x) + \log(x)} = x e^{\log(x_t)-\log(x)} = x e^{\hat{x}_t}
\end{align*}
It is very useful to do this transform, when you want to log-linearize your model,
  i.e.\ do a first-order Taylor approximation of the model equations with respect to logged variables (HAT VARIABLES \(\hat{x}_t\)).\footnote{%
Note that there are shortcuts to do a log-linearization without doing the exp transform for purely multiplicative or additive equations.
However, the exp transform always works, particularly for more difficult equations, so no need for learning the shortcuts.
}
\begin{itemize}

  \item Lets use this on equation~\eqref{eq:AS_B1}:
  \begin{align*}
  1 = \beta E_t \left[ {\left(\frac{c e^{\hat{c}_{t+1}}}{c e^{\hat{c}_t}}\right)}^{-\tau} \frac{1}{\gamma z e^{\hat{z}_{t+1}}} \frac{R e^{\hat{R}_t}}{\pi e^{\hat{\pi}_{t+1}}} \right]
  \end{align*}
  Note that equation~\eqref{eq:AS_B1} in steady-state is equal to \(1 = \beta \frac{1}{\gamma z}\frac{R}{\pi}\).
  So we can simplify the previous equation to get equation~\eqref{eq:AS_C1}:
  \begin{align*}
  1 &= \mathbb{E}_t \left[e^{-\tau \hat{c}_{t+1} + \tau \hat{c}_{t} + \hat{R}_{t} - \hat{z}_{t+1} - \hat{\pi}_{t+1} }\right]
  \end{align*}

  \item Log-Linearization: the first-order Taylor expansion of equation~\eqref{eq:AS_C1} with respect to the hat variables is:
  \begin{align*}
  1 = e^0 + e^0 E_t \left(-\tau \hat{c}_{t+1} + \tau \hat{c}_{t} + \hat{R}_{t} - \hat{z}_{t+1} - \hat{\pi}_{t+1}  \right)\\
  \Leftrightarrow
  0 = -\tau E_t \hat{c}_{t+1} + \tau \hat{c}_{t} + \hat{R}_{t} - E_t\hat{z}_{t+1} - E_t \hat{\pi}_{t+1}
  \end{align*}
  The first-order Taylor expansion of equation~\eqref{eq:AS_C3} with respect to the hat variables is:
  \begin{align*}
  e^0 + e^0\left(\hat{c}_t - \hat{y}_t\right) = e^0 - \frac{\phi \pi^2 g}{2} {(e^0-1)}^2 + e^0(-\hat{g}_t) - \frac{\phi \pi^2 g}{2} 2 \left(e^0-1\right)(\hat{\pi}_t)\\
  \Leftrightarrow \hat{c}_t = \hat{y}_t -\hat{g}_t
  \end{align*}
  Combining yields equation~\eqref{eq:AS_D1}:
  \begin{align*}
  \hat{y}_{t} &= \mathbb{E}_t \hat{y}_{t+1} + \hat{g}_{t} - E_t\hat{g}_{t+1} - \frac{1}{\tau} \left(\hat{R}_{t}- \mathbb{E}_t \hat{\pi}_{t+1} - \mathbb{E}_t \hat{z}_{t+1}\right)
  \end{align*}

\end{itemize}

\item Note that we also added the hat variables to make the equivalence between model variants even more evident:
\lstinputlisting[style=Matlab-editor,basicstyle=\mlttfamily\scriptsize,title=\lstname]{progs/dynare/an_schorfheide_nonlinear.mod}
\begin{verbatim}
THEORETICAL MOMENTS
VARIABLE         MEAN  STD. DEV.   VARIANCE
c              0.9487     0.0058     0.0000
z              1.0000     0.0069     0.0000
pie            1.0080     0.0070     0.0000
R              1.0156     0.0083     0.0001
Rstar          1.0156     0.0113     0.0001
y              1.1161     0.0225     0.0005
ystar          1.1161     0.0214     0.0005
g              1.1765     0.0226     0.0005
chat           0.0000     0.0061     0.0000
zhat           0.0000     0.0069     0.0000
piehat         0.0000     0.0069     0.0000
Rhat           0.0000     0.0082     0.0001
yhat           0.0000     0.0202     0.0004
ghat           0.0000     0.0192     0.0004
YGR            0.5000     1.1045     1.2199
INFL           3.2000     2.7787     7.7214
INT            6.2000     3.2712    10.7006
\end{verbatim}
Note that the moments and impulse response functions are exactly the same between model variants B, C and D
  when you compare the hat variables and measurement equations.

\item 
\lstinputlisting[style=Matlab-editor,basicstyle=\mlttfamily\scriptsize,title=\lstname]{progs/dynare/an_schorfheide_nonlinear_exp.mod}
\begin{verbatim}
THEORETICAL MOMENTS
VARIABLE         MEAN  STD. DEV.   VARIANCE
chat           0.0000     0.0061     0.0000
zhat           0.0000     0.0069     0.0000
piehat         0.0000     0.0069     0.0000
Rhat           0.0000     0.0082     0.0001
yhat           0.0000     0.0202     0.0004
ghat           0.0000     0.0192     0.0004
YGR            0.5000     1.1045     1.2199
INFL           3.2000     2.7787     7.7214
INT            6.2000     3.2712    10.7006
\end{verbatim}
Note that the moments and impulse response functions are exactly the same between model variants B, C and D
  when you compare the hat variables and measurement equations.

\item 
\lstinputlisting[style=Matlab-editor,basicstyle=\mlttfamily\scriptsize,title=\lstname]{progs/dynare/an_schorfheide_loglinear.mod}
\begin{verbatim}
THEORETICAL MOMENTS
VARIABLE         MEAN  STD. DEV.   VARIANCE
chat           0.0000     0.0061     0.0000
zhat           0.0000     0.0069     0.0000
piehat         0.0000     0.0069     0.0000
Rhat           0.0000     0.0082     0.0001
yhat           0.0000     0.0202     0.0004
ghat           0.0000     0.0192     0.0004
YGR            0.5000     1.1045     1.2199
INFL           3.2000     2.7787     7.7214
INT            6.2000     3.2712    10.7006
\end{verbatim}
Note that the moments and impulse response functions are exactly the same between model variants B, C and D
  when you compare the hat variables and measurement equations.

\item Note that all model variants yield the same model dynamics
  evident in the same moments and impulse response functions for the hat variables or the observable variables.
Therefore, one can simply add auxiliary variables for the log or the percentage deviation of a variable to a nonlinear mod file.
So there is no need to do a log-linearization on a model per se.
Moreover, log-linearizing the model equations by hand is often tedious and very error-prone.
By defining logged variables in the original nonlinear model equations and using first-order perturbation techniques,
  effectively Dynare does the linearization for you.
So in most cases, it is actually better to focus on the nonlinear stationarized model equations.
Also, you cannot do higher-order approximations on the log-linearized system.

On the other hand, log-linearized expressions often provide clear analytic intuition
  and reduce the model size which might be beneficial for e.g.\ advanced estimation techniques
  (for standard ones this is not really necessary).
\end{enumerate}