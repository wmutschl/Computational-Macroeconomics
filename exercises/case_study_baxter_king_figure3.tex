\section[Case Study Baxter and King (1993, AER): Macroeconomic Effects Of A Four-Year War]{Case Study Baxter and King (1993, AER): Macroeconomic Effects Of A Four-Year War\label{ex:CaseStudy.BaxterKing.Figure3}}

\begin{enumerate}
\item
Read the paper by \textcite{Baxter.King_1993_FiscalPolicyGeneral}
  and focus particularly on the model equations and analysis done in section IV.A.

\item
Write a Dynare mod file for the model and calibrate it according to
  the \emph{Benchmark Model with Basic Government Purchases} given in Table 1.
Compute the steady-state using an \texttt{initval} block.

\item
Replicate Figure 3.

\item
Do you think this simulation scenario reflects the essence of a war shock?

\end{enumerate}

\paragraph{Notes and hints}

\begin{itemize}
\item
According to footnote 3, the quantitative analysis corresponds to a model with labor-augmenting technical progress.
Calibrate \(\gamma_X\) such that the economy grows at 1.6\% per annum,
  a typical value used in e.g.\ \textcite{King.Plosser.Rebelo_1988_ProductionGrowthBusiness}.

\item
Pay attention to the different scales on the y-axes,
  i.e.\ commodity units, percent, and basis points.

\item
Skip the term structure variable in panel \emph{C.},
  as we will re-visit the term structure in RBC models later on.

\item
Use a perfect foresight simulation for computing the transition path
  from the initial steady-state (specified by an \texttt{initval} block)
  subject to a sequence of four basic government purchases shocks
  (specified by a \texttt{shocks} block).

\end{itemize}

\begin{solution}\textbf{Solution to \nameref{ex:CaseStudy.BaxterKing.Figure3}}
\ifDisplaySolutions%
\input{exercises/case_study_baxter_king_figure3_solution.tex}
\fi
\newpage
\end{solution}